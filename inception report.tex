\documentclass[12pt,a4paper,final]{article}
\usepackage[latin1]{inputenc}
\usepackage{amsmath}
\usepackage{amsfonts}
\usepackage{amssymb}
\usepackage{graphicx}


\usepackage[english]{babel}
\usepackage[latin1]{inputenc}
%\usepackage[leqno]{amsmath}

\usepackage{textcomp}
\usepackage{mathrsfs,euscript}
\usepackage{booktabs}
\usepackage{dcolumn}
\usepackage{epsfig}
\usepackage{upref}
\usepackage{paralist}
\usepackage{calc}
\usepackage{setspace}
%\usepackage[parfill]{parskip}
\usepackage[left=3.5cm,right=3.5cm,top=2cm,bottom=2cm]{geometry}
\usepackage[longnamesfirst]{natbib}
\bibliographystyle{plainnat}
%\usepackage{biblist}
%\usepackage{chicago}
%\bibliographystyle{chicago}
\usepackage{graphicx,float}
\usepackage{graphics}
\usepackage{multicol}
\usepackage{float}
\usepackage[normalem]{ulem}
\author{Eliab G. Luvanda}

\title{Construction of Business Cycle Indicators \\
\textit{Inception Report} \\
Submitted to \\
\textbf{Tanzania Revenue Authority (TRA)}}

\date{November 9, 2015}
\begin{document}

\maketitle

\newpage
\tableofcontents

\newpage
\section{Introduction}

\subsection{Background}
Business cycles are output fluctuations that involve movements in GDP overtime in alternating periods of expansion in economic activity (boom) and contraction in economic activity (recession). GDP/output being a base of tax revenue, business cycles are usually associated with fluctuations in tax revenue.  The upwardand downward swings in macroeconomic variables include GDP growth of domestic  and external economies, inflation, exchange rate, credit to the private sector, investment levels and more others.  These variables need to be monitored to detect their short and medium term influence on revenue collection.

\subsection{Objective of the assignment}
In view of the fact that business cycles have a significant impact on tax revenue, the Tanzania Revenue Authority (TRA) intends to carry out a study that will involve estimating business cycles and linking them to revenue collection performance in Tanzania.  The goal of carrying out this study is to enable TRA to monitor and track the developments in macroeconomic variables so that TRA can adopt measures that will cushion tax revenue from the adverse effects of swings in the economic activity (business cycles).

\subsection{Purpose of the inception report}
The main purpose of this inception report is to explain the consultant's understanding of the assignment as outlined by the specific objectives of the study, and to explain the methodology that the consultant will use to achieve the specific objectives of the study.

\subsection{Coverage of the inception report}
This inception report broadly covers the consultant's understanding of the terms of reference, the project's team, the methodology that will be used to accomplish specific objectives of the study, and the work plan.

\section{The assignment: clarification and definitions}
The main task for this assignment is estimate business cycles and and examining how the cycles influence tax revenue in Tanzania.  Business cycles in this case means fluctuation in output/GDP, the overall tax base, and also fluctuations in sectoral output/GDP, constituting bases for different taxes. In this case, tax revenue is defined in a way that excludes non-tax revenue.

\section{Terms of reference}

\subsection{Expected deliverables}

There are three major deleverables for this study. These are 

\begin{enumerate}
\item Inception report outlining the understanding of the terms of reference, and explaining in detail the metodology that will be used to attain the specific objectives of the study
\item Draft report of the findings of the study will be submitted to TRA for comments
\item Final report: After getting comments from TRA, the draft report will be revised to address TRA's comment, and the final report will be submitted to TRA.
\end{enumerate}

\subsection{Study coverage}
The study will cover all major tax categories. In terms of time coverage, in order to get more reliable results, the study is supposed cover a period which is reasonably long.  In this regard, availability of data will determine the maximum period that will be covered by this study.
 
\section{Understanding of the terms of reference}

The terms of reference are clearly stated in the specific objectives of the study. In this section, we explain our understanding of the specific objectives of the study.


\subsection{Specific objectives of the study}
In order to achieve the goal of the study, a number of specific objectives will be addressed.  These specific objectives include the following:

\begin{compactenum}[1)]
\item Examine the nature, main features and causes of business cycles in Tanzania and the associated reasons for their occurrence. Two (2) main tasks will have to be undertaken to address this specific objective. The first task will be to estimate and examine the trends and cyclic components of output (GDP), and other macroeconomic variables such as private consumption, investment, money stock, government expenditure, tax revenue, and the general price level/consumer price index (CPI) and the main characteristics of business cycles in Tanzania.  Specifically, the following two issues will have to be addressed; namely:
\begin{compactenum}[(i)]

\item	Using the relevant time series filters such as the Hodrick-Prescott (HP) filter, and Baxter-King (BK) filter to estimate the trend/permanent and cyclic components of macroeconomic variables (including GDP, sectoral GDP, private consumption, government expenditure, imports, inflation, exchange rate and tax revenue); and examining their patterns, such as the volatility of fluctuations/cyclic components;

\item b)	Identifying macroeconomic variables that are pro-cyclic (move together with output/GDP), and macroeconomic variables that are counter-cyclic (move in different directions with output/GDP) over the business cycle.
\end{compactenum}  

\item	Analyze economic sectors' behavior and  sectoral revenue data and determine the more responsive sectors in terms of upward and downward movements as well as their impact on revenue. This objective requires examining the behavior of economic sectors (such as agriculture, manufacturing, services, mining etc); and examine how the sectoral performance influence tax revenue.  More specifically, this objective will require the study to carry out the following:

\begin{compactenum}[(i)]
\item	Examine the sectoral performance (in terms of output) trends, and relative contribution of sectors to GDP;

\item	Examine the sectoral performance over the business cycle.  This will involve decomposing sectoral output into trend and cyclic components and examine the patterns of the components over time.  In addition the study will have to measure the volatility of sectoral output fluctuations over over the business cycle. 

\item Examine the extent to which tax revenue responds to movements in sectoral output in 2)(i) and 2) (ii) above. 

\end{compactenum}

\item 	Examine the extent to which tax revenue responds to both long term and cyclic movements of GDP/output. This will involve the following:

\begin{compactenum}[(i)]
\item	Examining how tax revenue responds to long-term movements in GDP and sectoral output/GDP. Basically, this will involve carrying out cointegration test to examine whether there is a long run equilbrium relation ship between output and tax revenue.

\item Examining how tax revenue responds to short-term movements in GDP and sectoral output/GDP.  
\end{compactenum}

\item Prepare a model report on Tanzania business cycles linking with tax collection.  This will involve coming up with a report, which among other things, will include a model linking tax revenue with business cycles.
 
\end{compactenum}

\section{Project team and management}

The project team will comprise of three economists, one senior member among them being the team leader. The project team will be composed of the following:

\begin{enumerate}
\item Dr. Eliab G Luvanda: Team leader
\item Mr. Hamza Mkai: Team member
\item Mr. Elinnema Kisanga: Team member
\end{enumerate}

The specific roles of the team members are shown in Table 1 below.

\begin{table}[h]
\centering
\caption{Team members and their specific roles}
\begin{tabular}{|l|l|}
\hline Name  & Role \\ 
\hline Dr. Eliab Luvanda  & Team leader, Data analysis, report write up  \\ 
\hline Mr. Hamza Mkai  & Data analysis \\ 
\hline Mr. Elinnema Kisanga & Data analysis \\ 
\hline 
\end{tabular} 
\end{table}

Three members of staff in the Directorate of Research of the Tanzania Revenue Authority will also be in the team.  The role of these members will be to provide data, analyze data, and to provide qualitative information which will be needed to compliment quantitative information.

One important reason for inclusion of the TRA members of staff is capacity building.  The consultant will work together with the TRA members of staff, and in the process the TRA members of staff will be able to learn all the technical aspects of the project, such as the softaware, estimation of business cycles, and econometric models linking tax revenue with the relevant tax bases.
\section{Methodology}
Given the specific objectives of the study outlined in the previous section, this section on methodology describes the methods this study will use to address the specific objectives of the study.

\subsection{Literature Review}

Both theoretical and empirical literature related to business cycles and the behaviour of tax revenue over the business cycle will be reviewed.  The main aim of this review is to explore the theory behind business cycles in relation to tax revenue and guide the choice of analytical techniques for carrying out the assignment.

\subsection{Data Collection}

Secondary data will be used. Time series data for macroeconomic variables (including tax revenue) will be obtained from official publications (database kept) by the Bank of Tanzania (BOT), the National Bureau of Statistics (NBS), and Tanzania Revenue Authority (TRA).  The International Financial Statistics (IFS) will be another source of time series data for the variables.  Qualitative information will be obtained from the publications; and if necessary, from interviews with officials from the institutions. 

\subsection{Data Analysis}

Data analysis will be carried out to estimate the trend and cyclic components of GDP/output other macroeconomic variables such as tax revenue, inflation, government expenditure and money supply. Data analysis will also involve computation of correlation coefficients, cointegration test and estimation of econometric models to examine the relationsip between output and tax revenue.

\subsubsection{Estimation of Trend and Cylic Components of Macroeconomic Variables}

Standard filters such as the Hodrick ? Prescott (HP) and Baxter King (BK) filters are commonly used to estimate the trend and cyclic components of the variables.  This study will use the Hodrick - Prescott filter, the most commonly used filter in the business cycle studies.

\subsubsection*{Decomposition of Time Series into Trend and Cyclical Components}

A macroeconomic time series   is composed of two main components: (1) the permanent (or secular) component $y_t^p$ , and (2) the transitory component  $y_t^c$. Thus, a macroeconomic variable can be represented by the following simple model:

\[ y_t = y_t^p + y_t^c \]

where $y_t$  is logarithm of actual observation, $y_t^p$  is a trend or secular component and $y_t^c$  represents deviations from trend or a cyclical component.  Stochastic detrending methods can be used to estimate the trend and cyclic components of macroeconomic variables. The most popular among the stochastic detrending methods is the Hodrick Prescott  (hereafter referred to as HP) (1980) filter.  The other and more recent is the Baxter--King (1995) filter. The HP trend for a variable $y_t$ (in logarithmic form) is found by minimizing the following function:

\[ \sum_{t=1}^T\left( y_t - y_t^{trend}\right) ^2 + \lambda \sum_{t=2}^{T-1} \left[ \left( y_{t+1}^{trend} - y_t^{trend}\right) -\left( y_t^{trend} - y_{t-1}^{trend}\right)  \right] \]

The idea behind this equation is to minimize the sum of two components.  The first component is the sum of squares of the deviation of the actual value of a variable from its trend value.  The second is the sum of squares of changes in the trend growth.  A series of trend values that gives the minimum value of the equation is the HP trend or filter.
A parameter, $\lambda$, controls the degree of smoothness in the HP trend.  On the one hand, a smaller value of $\lambda$ reduces the importance of change in trend growth.  For $\lambda = 0$, a minimum value of the equation is given by selecting the trend value equal to the actual value of the variable.  In this case, there cannot be deviations from trend.  On the other hand, a very high $\lambda$ diminishes the significance of the first component.  In the limit, a smooth deterministic trend yields the minimum of the total sum, and the cyclical component of a variable will be large. 

To get a `reasonable' variable trend, one must choose an intermediate value of $\lambda$.  Hodrick and Prescott (1980) recommend $\lambda = 1600$ for quarterly data, and $\lambda = 100$ for annual data.  However, Ravn et al (1997) recommend $\lambda = 6.75$ for annual data. Once a trend component, has been estimated, the cyclical component of the variable can be obtained by subtracting the trend value from the actual observations of the variable; that is, $y_t^c = y_t - y_t^p$. 

Once the trend and cyclic components have been estimated, they are usually plotted on a graph from which one can study the pattern of long term movement of the variable, and the pattern of cyclic movement/fluctuations of the variable.

Preliminary estimates of cyclic and trend components for some of the macroeconomic variables are attached as an Appendix.

\subsubsection{Computation of Correlation Coefficients and Variances}

Once the cyclic components of the variables have been estimated, as explained in the preceding section, correlation coefficients between the cyclic component of output/GDP (as a reference variable) and cyclic components of other macroeconomic variables (including tax revenue).  The correlation coefficients (negative or positive) will help in determining whether the other macroeconomic variables are are procyclical (the cyclic component of the variable move in the same direction as the cyclic component of output/GDP); or whether the other macroeconomic variables are counter-cyclical (the cyclic component of the variable move in the different relative to the cyclic component of output/GDP). Variances of the cyclical components of the variables will be computed to measure the volatitilty of fluctuations of the economic variables over the business cycle.

\subsection{Econometric Analysis}

Econometric analysis will be employed to examine the main factors (monetary and real factors) that drive the busines cycles in Tanzania; and to examine long term movements in output do influence tax revenue (categories of tax revenue), and whether short term movements over the business cycle in output do influence tax revenue.

\subsubsection{Vector Autoregression (VAR) Model} 

A simple vector autoregression (VAR) model will be estimated and used to examine the relative importance of monetary and real factors in explaining the variations in output. The following VAR model will be estimated:

\[ Y_t = \Phi_0 +\Phi_1 Y_{t-1} + \ldots + \Phi_p Y_{t-p} + \varepsilon_t \]

where $Y_t$ is a vector of output, monetary and real factors, $\Phi_0$ is a vector of constants, $\Phi_i$ ($i=1,\ldots,p$) are coefficient matrices; and $\varepsilon_t$ is a vector of error terms that are assumed to be independent and identically distributed.

Once the VAR model has been estimated, ganger causality test will be used to examine whether monetary or real factors do influence output. The variance decompostion will be used to examine the relative importance monetary and real factors in explaining the business cycles.

\subsubsection{Cointegration Test}

Cointegration test will be carried out to examine whether there is a long run equilibrium relationship between GDP/output/sectoral GDP and tax revenue/categories of tax revenue. In other words this test will help in determining whether tax revenue respond to long term movement in output/GDP. The output/GDP coefficient in the cointegrating equation will be the measure of long run tax elasticity

\subsubsection{Estimation of Simple Econometric Models}

Simple econometric models will be estimated to examine how tax revenue responds to short run movements in output/GDP over the business cycle. The following simple model will be estimated:

\[ T_t^c = \alpha + \beta Y_t^c + \varepsilon_t \]

where $T_t^c$ is cyclical component of cyclic tax revenue, and $Y_t^c$ is cyclic component of output/GDP. In this case, cyclic components of different tax revenue will be estimated as a function of cyclic component of the tax base.

\subsection{Software package}
\textbf{R}, an open source programming environment,  will be used for data analysis. 

\section{Work plan}

\subsection{Deliverables and duration}
Table 2 below presents the major deliverables and duration

\begin{table}[h]
\centering
\caption{Deliverables and duration}
\begin{tabular}{|l|l|c|}
\hline Serial No.  & Task   & Duration (days)  \\ 
\hline 1.  & Inception report preparation & 6 \\ 
\hline 2.  & Draft report preparation & 18 \\ 
\hline 3. & Final Rreport preparation & 6 \\ 
\hline  & Total man-days  & 30 \\ 
\hline 
\end{tabular} 
\end{table}

\subsection{Assignment completion}
Once the final report has been completed, it will be submitted to the client, Tanzania Revenue Authority.
\end{document}